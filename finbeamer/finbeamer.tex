\documentclass[aspectratio=169]{beamer}

\usepackage[T1, T2A]{fontenc}
\usepackage[english, serbianc]{babel}
\usepackage{tikz}

\definecolor{UniRed}{HTML}{d40019}
\definecolor{UniGreen}{HTML}{3c671d}
\definecolor{UniBlue}{HTML}{0062aa}
\definecolor{UniPurple}{HTML}{69227f}
\definecolor{UniAmber}{HTML}{fcb900}
\definecolor{UniBlack}{HTML}{1b1918}
\definecolor{FinCyan}{HTML}{6fdaf8}
\definecolor{FinGray}{HTML}{c4c5c7}
\definecolor{FinBlack}{HTML}{201d1e}
\definecolor{TocSubsection}{HTML}{1F4D80}

\usetheme{Warsaw}
\usecolortheme{structure}
\setbeamercolor{structure}{fg=UniBlue}
\setbeamercolor{subsection in toc}{fg=TocSubsection}
% \setbeamertemplate{description item}{\makebox[3em][l]{\insertdescriptionitem}}
\setbeamertemplate{headline}{}
\setbeamertemplate{navigation symbols}{}
\setbeamertemplate{section in toc}[square]

\newcommand\numbered{\setbeamertemplate{footline}{\vspace{-12pt}\hfill\usebeamercolor[fg]{page number in head/foot}\usebeamerfont{page number in head/foot}\insertframenumber\,/\,\inserttotalframenumber\kern2pt\vskip2pt}}
\newcommand\samenumbered{\setbeamertemplate{footline}{\vspace{-12pt}\hfill\usebeamercolor[fg]{page number in head/foot}\usebeamerfont{page number in head/foot}\insertframenumber\,/\,\inserttotalframenumber\kern2pt\vskip2pt}\addtocounter{framenumber}{-1}}
\newcommand\unnumbered{\setbeamertemplate{footline}{}\addtocounter{framenumber}{-1}}

% Избор логотипа на насловној страни:
% \titlelogouni --- универзитетски лого
% \titlelogofin --- факултетски лого
% \titlelogoboth --- оба логотипа
\newcommand{\titlelogofin}{\def\titlelogo{\begin{figure}\includegraphics[height=55pt]{./slike/grb/finkg_logo}\end{figure}}}
\newcommand{\titlelogouni}{\def\titlelogo{\begin{figure}\includegraphics[height=55pt]{./slike/grb/unikg_logo}\end{figure}}}
\newcommand{\titlelogoboth}{\def\titlelogo{\begin{figure}\includegraphics[height=55pt]{./slike/grb/unikg_logo}\qquad\includegraphics[height=55pt]{./slike/grb/finkg_logo}\end{figure}}}

\titlelogoboth

% Избор логотипа на подножју слајдова:
% \footlogouni --- универзитетски лого
% \footlogofin --- факултетски лого
% \footlogo{...} --- произвољна слика
\newcommand{\footlogo}[1]{\setbeamertemplate{background canvas}{%
\ifnum\insertframenumber>0
    \begin{tikzpicture}[overlay, remember picture]
        \node[anchor=south west] at (current page.south west) {\includegraphics[width=20pt]{#1}};
    \end{tikzpicture}%
\fi}}
\newcommand{\footlogouni}{\footlogo{slike/grb/unikg_logo.pdf}}
\newcommand{\footlogofin}{\footlogo{slike/grb/finkg_logo.pdf}}

\footlogouni


\title{\LARGE{Наслов презентације}}

\author{Име и презиме\texorpdfstring{\vspace{-1em}}{}}%
\institute{\textbf{Катедра за електротехнику и рачунарство}\\Факултет инжењерских наука Универзитета у Крагујевцу%
\vspace{7pt}\titlelogo\vspace{-9pt}}
\date{Крагујевац, \today}

% Приказивање садржаја између поглавља
% \AtBeginSection[]{%
%     \begin{frame}{\contentsname}
%         \tableofcontents[currentsection, hideallsubsections]
%     \end{frame}%
% }

\begin{document}
\numbered

{\unnumbered
\begin{frame}
   \titlepage
\end{frame}}

\section*{Садржај}
\begin{frame}{\secname}
    \tableofcontents[hideallsubsections]
\end{frame}

\section{Поглавље}
\begin{frame}{\secname}
    \begin{itemize}
        \item Прва ставка
        \item Друга ставка
    \end{itemize}
    \pause
    
    \medskip Пример:
    \begin{enumerate}
        \item Трећа ставка
        \item Четврта ставка
    \end{enumerate}
\end{frame}

\subsection{Потпоглавље}
\begin{frame}{\secname: \subsecname}
    \begin{columns}
        \begin{column}{0.5\linewidth}
            \begin{itemize}
                \item Пета ставка
                \item Шеста ставка
            \end{itemize}
        \end{column}
        \begin{column}{0.5\linewidth}
            \begin{enumerate}
                \item Седма ставка
                \item Осма ставка
            \end{enumerate}
        \end{column}
    \end{columns}
    \pause

    \bigskip
    \begin{description}
        \item[Девета ставка] Опис девете ставке
        \item[Десета ставка] Опис десете ставке
    \end{description}
\end{frame}

\section{Резиме}
\begin{frame}{\secname}
    \begin{itemize}
        \item Шаблон
        \item Поглавља и потпоглавља
        \item Енумерације
        \item Паузе
        \item Колоне
    \end{itemize}
\end{frame}

\section*{Навигација за питања}
% Са "sections=..." се одређује која поглавља се приказују
{\setbeamertemplate{footline}{}\begin{frame}[noframenumbering]{\secname}
    \begin{columns}[t]
        \begin{column}{0.5\linewidth}
            \tableofcontents[sections={2-4}]
        \end{column}
        \begin{column}{0.5\linewidth}
            \tableofcontents[sections={5-}]
        \end{column}
    \end{columns}
\end{frame}}

\end{document}
